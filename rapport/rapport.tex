\documentclass{article}
\usepackage[utf8]{inputenc}

\usepackage{graphicx}
\usepackage{color}
\usepackage{float}
\usepackage[pdf]{graphviz}

\usepackage{geometry}
\geometry{hmargin=2.5cm,vmargin=1.5cm}

\begin{document}

\begin{figure}[t]
\centering
\includegraphics[width=5cm]{inp_n7.png}
\end{figure}

\title{\vspace{4cm} \textbf{Projet de programmation fonctionnelle et de traduction des langages}}
\author{MDAA Saad | El Bouzekraoui Younes}
\date{\vspace{7cm} Département Sciences du Numérique - Deuxième année \\
2020-2021 }

\maketitle

\newpage
\tableofcontents

\newpage
%%%%%%%%%%%%%%%%%%%%%%%%%%%%%%%%%%%%%%%%%%%%%%%%%%%%%%%%%%%%%%%%%%
\section{Introduction}
Le but de ce projet est de ajouter les extensions du language Rat  réalisé en TP de traduction des langages en Ocaml afin de traiter les nouvelles 
constructions 
\begin{itemize}
    \item pointeurs
    \item surcharge des fonctions
    \item types énumérés
    \item switch/case
\end{itemize}
\section{Pointeurs}
\subsection{Modifications de la grammaire}
On ajoute les règles suivantes :
\begin{itemize}
    \item TYPE $\rightarrow$ TYPE *
    \item E $\rightarrow$ A
    \item A $\rightarrow$ * A
    \item A $\rightarrow$ id
    \item E $\rightarrow$ null
    \item E $\rightarrow$ new TYPE
    \item E $\rightarrow$ \& id
    \item I $\rightarrow$ A = E
\end{itemize}
Afin de traiter l'extension des pointeurs on doit ajouter un nouveau type dans \textbf{type.ml} : Pointeur of typ,
un pointeur sur null a un type de Pointeur(Undefined)
\subsection{Jugement de typage}
\begin{equation}
    \frac{\sigma \vdash a : Pointeur(\tau)}{\sigma \vdash * a : \tau}
\end{equation}
\begin{equation}
    \frac{\sigma \vdash a :\tau}{\sigma \vdash \& a : Pointeur(\tau)}
\end{equation}
\begin{equation}
    \sigma \vdash null : Pointeur(Undefined)
\end{equation}
\begin{equation}
    \frac{\sigma \vdash a :\tau}{\sigma \vdash new a : Pointeur(\tau)}
\end{equation}
\section{Surcharge des fonctions}
\section{Types énumérés}
\subsection{Modifications de la grammaire}
On ajoute les règles suivantes :
\begin{itemize}
    \item Main $\rightarrow$ ENUMS PROG
    \item ENUMS $\rightarrow$ $\Lambda$
    \item ENUMS $\rightarrow$ ENUM ENUMS
    \item ENUM $\rightarrow$ enum tid {IDS}
    \item IDS $\rightarrow$ tid
    \item IDS $\rightarrow$ tid, IDS
    \item TYPE $\rightarrow$ tid
    \item E $\rightarrow$ tid
\end{itemize}
\subsection{Jugement de typage}
\begin{equation}
    \frac{\sigma \vdash E_1 : Enum(n), \sigma \vdash E_2 : Enum(n)}{\sigma \vdash (E_1=E_2) : bool}
\end{equation}

\section{Switch/case}
\subsection{Modifications de la grammaire}
On ajoute les règles suivantes :
\begin{itemize}
    \item I $\rightarrow$ switch (E) {LCase}
    \item LCase $\rightarrow$ Case LCase
    \item LCase $\rightarrow$ $\Lambda$
    \item Case $\rightarrow$ case tid : IS B
    \item Case $\rightarrow$ case entier : IS B
    \item Case $\rightarrow$ case true : IS B
    \item Case $\rightarrow$ case false : IS B
    \item Case $\rightarrow$ default : IS B
    \item B $\rightarrow$ $\Lambda$
    \item B $\rightarrow$ break;
\end{itemize}
\subsection{Jugement de typage}
\begin{equation}
    \frac{\sigma \vdash E : \tau, \sigma \vdash E_1 : \tau, \sigma \vdash Bloc_1: void, \sigma \vdash default : Undefined, \sigma \vdash Bloc_3 : void}
        {\sigma \vdash switch (E) case E_1 Bloc_1 ... default Bloc_3 : void,[]}
\end{equation}

\end{document}