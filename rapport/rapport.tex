\documentclass{article}
\usepackage[utf8]{inputenc}

\usepackage{graphicx}
\usepackage{color}
\usepackage{float}
\usepackage[pdf]{graphviz}

\usepackage{geometry}
\geometry{hmargin=2.5cm,vmargin=1.5cm}

\begin{document}

\begin{figure}[t]
\centering
\includegraphics[width=5cm]{inp_n7.png}
\end{figure}

\title{\vspace{4cm} \textbf{Projet de programmation fonctionnelle et de traduction des langages}}
\author{MDAA Saad | El Bouzekraoui Younes}
\date{\vspace{7cm} Département Sciences du Numérique - Deuxième année \\
2020-2021 }

\maketitle

\newpage
\tableofcontents

\newpage
%%%%%%%%%%%%%%%%%%%%%%%%%%%%%%%%%%%%%%%%%%%%%%%%%%%%%%%%%%%%%%%%%%
\section{Introduction}
Le but de ce projet est de ajouter les extensions du language Rat  réalisé en TP de traduction des langages en Ocaml afin de traiter les nouvelles 
constructions 
\begin{itemize}
    \item pointeurs
    \item surcharge des fonctions
    \item types énumérés
    \item switch/case
\end{itemize}
\section{Pointeurs}
\section{Surcharge des fonctions}
\section{Types énumérés}
\section{Switch/case}

\end{document}